%加载并设置宏包
%广西大学物理学院2021级硕士王信哲制作
\ctexset{
	bibname = 参考文献,
	contentsname=\heiti\zihao{3} 目\ 录,
	section={
		name={第,章},number=\chinese{section},
		format=\heiti\bfseries\centering\zihao{3}\linespread{1.25},
		beforeskip=20pt,%18ex plus 0.2ex minus .2ex,
		afterskip=20pt%18ex plus 0.2ex minus .2ex
	},
	subsection={
		format=\heiti\bfseries\raggedright\zihao{4}\linespread{1.25},
		beforeskip=8.75pt,%18ex plus 0.1ex minus .1ex,
		afterskip=8.75pt%18ex plus 0.1ex minus .1ex
	},
	subsubsection={
		format=\heiti\bfseries\raggedright\zihao{-4}\linespread{1.25},
		beforeskip=0pt,
		afterskip=0pt
	},
	appendix/name=附录
}
% \usepackage[UTF-8]{ctex}%設置中文

\usepackage{tocloft}
\renewcommand{\cfttoctitlefont}{\hfill} % 将目录标题设置为三号字、黑体
\renewcommand{\cftaftertoctitle}{\hfill} % 将目录标题居中
\renewcommand{\cftsecleader}{\cftdotfill{\cftdotsep}}%让一级标题的章节和页面之间也有点线连接
\renewcommand{\cftsecfont}{\normalfont}%取消section字体在目录中加粗
\renewcommand{\cftsecpagefont}{\normalfont}%取消section在目录中对应的页码加粗
\setlength{\cftbeforesecskip}{0pt}
\renewcommand{\cftdotsep}{1}


\setCJKmainfont{STSong}%设置全局中文字体
\usepackage{fontspec}%设置英文
\setmainfont{Times New Roman} % 设置英文为Times New Roman
\usepackage{amsmath,amssymb,amsthm}
\let\hbaroriginal\hbar
\AtBeginDocument{\let\hbar\hbaroriginal}
%\usepackage{unicode-math}
%\setmathfont{Latin Modern Math}

\usepackage{geometry}
\geometry{papersize={21cm,29.7cm}}
\geometry{left=2.8cm,right=2.5cm,top=2.8cm,bottom=2.2cm}
\usepackage{mathrsfs}
\usepackage{wasysym}
\usepackage{braket}
\usepackage{indentfirst}
\usepackage{cases}
\usepackage{bm}
\usepackage{graphicx}
\usepackage{subcaption}
\usepackage{longtable,booktabs}
\usepackage{color}
\usepackage[colorlinks,linkcolor=blue,anchorcolor=blue,citecolor=blue]{hyperref}%#006eb2
%\hypersetup{
	%	colorlinks=true,
	%	linkcolor=[rgb]{0,0.43,0.70},
	%	citecolor=[rgb]{0,0.43,0.70},
	%	filecolor=[rgb]{0,0.43,0.70},
	%	urlcolor=[rgb]{0,0.43,0.70}
	%}
\usepackage{cleveref}   %可以调用 \Cref{fig:xxx}或者 \ref{fig:xxx}
\usepackage{pdfpages}
\usepackage{float}
\usepackage{booktabs}
\usepackage{fancyhdr}%設置頁眉

\usepackage{setspace}%设置行间距
\setstretch{1.25}

\usepackage[nottoc]{tocbibind}%[nottoc,numbib]
%\usepackage{gbt7714}
\usepackage{gbt7714-gxu}
\bibliographystyle{gxu}
\usepackage{natbib}
\setlength{\bibsep}{0pt}
%\bibliographystyle{gbt7714-numerical}

\usepackage{appendix}


\usepackage{booktabs}
\usepackage{lipsum} % 用于生成随机文本
\graphicspath{{图/}{./}}
\usepackage[justification=centering]{caption}
\usepackage{bicaption}
\captionsetup[figure][bi]{labelfont=bf, textfont=bf}
\captionsetup[figure][bi-second]{name=Figure,labelfont=bf,textfont=bf}
\captionsetup[table][bi]{labelfont=bf, textfont=bf}
\captionsetup[table][bi-second]{name=Table,labelfont=bf,textfont=bf}

%设置图表编号格式为{章节号}-{章节内序号}
\usepackage{chngcntr}
\counterwithin{figure}{section}
\counterwithin{table}{section}
\renewcommand{\thefigure}{\thesection-\arabic{figure}}
\renewcommand{\thetable}{\thesection-\arabic{table}}

\usepackage{pdfpages}

% 定义中文摘要环境
\newenvironment{cnabstract}{
	\par\small
	\noindent\mbox{}\hfill{\bfseries \cnabstractname}\hfill\mbox{}
	\par\vskip 2.5ex
}{\par\vskip 2.5ex}

% 定义英文摘要环境
\newenvironment{enabstract}{
	\par\small
	\noindent\mbox{}\hfill{\bfseries \enabstractname}\hfill\mbox{}
	\par\vskip 2.5ex
}{\par\vskip 2.5ex}

\newcommand{\cnabstractname}{\heiti \fontsize{16pt}{20pt}\textbf{摘\ 要}} % 中文摘要标题
\newcommand{\enabstractname}{\fontsize{16pt}{20pt}\textbf{ABSTRACT}} % 英文摘要标题
