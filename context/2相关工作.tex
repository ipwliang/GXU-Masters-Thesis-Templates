\section{相关理论基础\label{相关工作}}

\subsection{图像去雾还原的理论基础与算法}

相关工作二级标题的正文,相关工作二级标题的正文,相关工作二级标题的正文,相关工作二级标题的正文,相关工作二级标题的正文。

\subsubsection{雾天图像成像原理与大气散射模型}

相关工作三级标题的正文,相关工作三级标题的正文,相关工作三级标题的正文,相关工作三级标题的正文,相关工作三级标题的正文。

\subsubsection{传统的图像去雾算法相关研究}

\subsubsection{基于深度学习的图像去雾算法相关研究}


\subsection{目标检测的理论基础与算法}

相关工作二级标题的正文,相关工作二级标题的正文,相关工作二级标题的正文,相关工作二级标题的正文,相关工作二级标题的正文。

\subsubsection{机器学习目标检测算法}

相关工作三级标题的正文,相关工作三级标题的正文,相关工作三级标题的正文,相关工作三级标题的正文,相关工作三级标题的正文。

\subsubsection{双阶段目标检测算法}


\subsubsection{单阶段目标检测算法}



\subsection{图像去雾还原算法的评价指标}

\subsubsection{成对图像评价指标}

\subsubsection{单图像评价指标}

\subsection{目标检测算法的评价指标}

\subsubsection{混淆矩阵}

\subsubsection{平均精度}

\subsubsection{参数量和计算量}


\subsection{本章小结}





% 公式的示例\footnote{这是一个脚注的示例}:
\begin{equation}
    \frac{\hbar c}{e^2}=137
\end{equation}

\begin{subnumcases}{\label{eq:相关工作|对称的麦克斯韦方程组}}
    \text{d}^*\!\bm{F}=4\pi^*\!\!\bm{J}\\
    \text{d}\bm{F}=4\pi^*\!\!\hat{\bm{J}}
\end{subnumcases}