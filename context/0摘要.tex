%% 中文摘要
\section*{\ArticleTitle}
\begin{cnabstract}\addcontentsline{toc}{section}{摘\ 要}
\fontsize{14pt}{17.5pt}\selectfont %设置字体为四号,行间距1.25倍

本研究聚焦于雾天环境下交通小目标检测这一关键课题,旨在为智能交通系统和无人技术在复杂气象条件下的应用提供高效、鲁棒的解决方案。随着智能交通系统的快速发展,交通物体的检测与识别对于保障道路安全和优化交通管理具有不可替代的重要作用。然而,雾天等恶劣天气条件严重干扰了无人机拍摄图像的质量,导致基于良好天气图像训练的检测模型性能大幅下降,尤其对于小目标检测而言,挑战更为严峻。

雾天环境下交通小目标检测面临着多重问题与挑战。雾气的弥散作用导致图像能见度降低、对比度下降,目标轮廓模糊不清,边界难以辨认,这使得基于良好天气图像训练的检测模型难以适应,准确识别目标的能力受到极大限制。小目标在图像中所占像素比例极低,传统检测算法受限于分辨率和信息丢失问题,难以有效捕捉和识别这些小目标。在雾天条件下,小目标检测的难度被进一步放大,去雾过程可能导致目标细节的丢失,影响检测精度;不同去雾算法在不同雾天场景下的适用性和稳定性存在差异;去雾与检测算法的融合优化尚未达到理想状态,两者的协同作用未能充分发挥,导致整体检测系统的性能提升有限。

为应对上述挑战,本研究提出了一系列创新性的方法。在目标检测方面,以 YOLOv11 算法为基础,提出了改进的 EX-YOLO 算法。通过深入分析 YOLOv11 网络结构,发现其在交通小目标检测中存在浅层特征图分辨率不足、跨尺度特征融合效率低等问题。为此,引入了改进的特征融合模块 SPPC,融合 SPPF 模块的多尺度特征提取能力和 CAM 的特征增强能力,显著提升了对小目标的特征捕获能力。同时,采用轻量化的卷积模块 DBSS 替代原始模块,降低了计算复杂度,使其更适用于资源受限设备。此外,引入 NWD 损失函数与 DIoU 损失函数相结合,提高了对小目标定位的精准度。在图像去雾方面,基于 CycleGAN 网络提出了改进的 CGANFormer 算法。将 Transformer 模块与 CycleGAN 的生成器网络相结合,利用其自注意力机制捕捉图像全局依赖关系,突破了传统卷积局部感受野的限制,能够更精准地还原雾天图像中的细节信息,如物体边缘和纹理。同时,采用局部 - 全局结合的判别器架构,局部判别器关注图像细节特征,全局判别器把控图像宏观特征,提升了去雾图像的整体质量和视觉效果。

本研究方法在多个数据集上进行了全面验证,取得了显著的效果。EX-YOLO 算法在保持较高检测精度的同时大幅降低了计算量,与 YOLOv11s、YOLOv10s 和 YOLOv9s 等模型相比,在 mAP、P、R 等关键指标上均展现出优异的综合性能,尤其在小目标检测任务中具有很强的竞争力,有效平衡了精度与效率,满足了无人机等有限设备环境下的实际应用需求。CGANFormer算法与AOD-Net、FFA-Net、CGA-Net等网络模型对比,表现出更有效的去雾性能。CGANFormer 与 EX-YOLO 算法的结合在雾天图像去雾和目标检测联合实验中表现出色,在 FOG-TT100K 和 FOG-VisDrone 数据集上的精确度、召回率和 mAP 等指标上均取得显著提升,证明了该算法在无人机等资源受限设备上进行实时目标检测的可行性,有效解决了雾天环境下交通目标检测的难点问题。

本研究在雾天环境下交通小目标检测领域具有重要的意义。一方面,通过改进图像去雾和目标检测算法,为智能交通系统在恶劣天气条件下的稳定运行提供了技术保障,有助于提高道路安全和交通管理效率。另一方面,研究成果推动了计算机视觉技术在无人领域的应用,为无人机在复杂环境中的交通监测、目标识别等任务提供了更可靠的解决方案,促进了无人技术的进一步发展。此外,本研究提出的算法在保持高性能的同时降低了计算资源需求,有利于降低智能交通系统和无人设备的硬件成本,提高其实用性和可推广性。未来,随着相关技术的不断完善和优化,本研究成果有望在更广泛的交通场景和气象条件下发挥更大的作用,为构建更加智能、安全、高效的交通体系奠定坚实的基础。
\\
\\
\heiti 关键词:
\songti 无人机\ \ 雾天目标检测\ \ 改进YOLO算法\ \ 循环对抗生成网络\ \ Transformer模块

\end{cnabstract}
\pagebreak


%% 英文摘要
\section*{\ArticleTitleEn}
\begin{enabstract}\addcontentsline{toc}{section}{ABSTRACT}
\fontsize{14pt}{17.5pt}\selectfont %设置字体为四号,行间距1.25倍

This research focuses on the key topic of detecting small traffic targets in foggy environments, aiming to provide efficient and robust solutions for the application of intelligent transportation systems and unmanned technologies under complex meteorological conditions. With the rapid development of intelligent transportation systems, the detection and identification of traffic objects plays an irreplaceable and important role in ensuring road safety and optimizing traffic management. However, bad weather conditions such as foggy days have seriously interfered with the quality of the images taken by drones, resulting in a significant decline in the performance of detection models based on good weather image training, especially for small target detection, which is more serious.

Traffic small target detection faces multiple problems and challenges in a foggy environment. The diffusion effect of fog leads to reduced image visibility, reduced contrast, blurred target contours, and unrecognizable boundaries, which makes it difficult for detection models based on good weather image training to adapt, and the ability to accurately identify targets is greatly limited. The pixel ratio of small targets in the image is extremely low. Traditional detection algorithms are limited by resolution and information loss problems, and it is difficult to effectively capture and identify these small targets. Under the condition of foggy weather, the difficulty of small target detection is further amplified. The defogging process may lead to the loss of target details, affecting the detection accuracy. There are differences in the applicability and stability of different fogging algorithms in different foggy weather scenarios. The integration optimization of fogging and detection algorithms has not reached the ideal state, and the synergy of the two Failure to give full play has led to a limited improvement in the performance of the overall detection system.

In order to meet the above challenges, this study puts forward a series of innovative methods. In terms of target detection, an improved EX-YOLO algorithm is proposed based on the YOLOv11 algorithm. Through in-depth analysis of the YOLOv11 network structure, it is found that it has problems such as insufficient resolution of shallow feature map and low cross-scale feature fusion efficiency in traffic small target detection. To this end, an improved feature fusion module SPPC has been introduced to integrate the multi-scale feature extraction ability of the SPPF module and the feature enhancement ability of CAM, which significantly improves the feature capture ability for small targets. At the same time, the use of lightweight convolutional module DBSS to replace the original module reduces the complexity of computing and makes it more suitable for resource-limited devices. In addition, the combination of the NWD loss function and the DIoU loss function has improved the accuracy of small target positioning. In terms of image defogging, an improved CGANFormer algorithm is proposed based on the CycleGAN network. Combining the Transformer module with CycleGAN's generator network, it uses its self-attention mechanism to capture the global dependency of images, breaks through the limitations of traditional convolutional local sensing fields, and can more accurately restore detailed information in the foggy images, such as object edges and textures. At the same time, the local-ground combined discister architecture is adopted. The local discisiser focuses on the detailed characteristics of the image, and the global discisifier controls the macro characteristics of the image, which improves the overall quality and visual effect of the defogging image.

This research method has been comprehensively verified on multiple data sets and achieved remarkable results. The EX-YOLO algorithm greatly reduces the calculation amount while maintaining high detection accuracy. Compared with benchmark models such as YOLOv11s, YOLOv10s and YOLOv9s, it shows excellent comprehensive performance in key indicators such as mAP, P and R, especially in It has strong competitiveness in small target detection tasks, effectively balances accuracy and efficiency, and meets the actual application needs of limited equipment environments such as unmanned aerial vehicles. The combination of CGANFormer and EX-YOLO algorithm performed well in the joint experiments of foggy image defogging and target detection, and the accuracy, recall rate and mAP on the FOG-TT100K and FOG-VisDrone data sets are all taken. It has been significantly improved, and at the same time, the computing volume has been significantly reduced, showing a low computational complexity, which proves the feasibility of the algorithm for real-time target detection on resource-limited equipment such as drones, and effectively solves the difficult problem of traffic target detection in a foggy environment.

This research is of great significance in the field of traffic small target detection in a foggy environment. On the one hand, by improving image defogging and target detection algorithms, it provides technical guarantee for the stable operation of intelligent traffic systems under bad weather conditions, which is conducive to improving road safety and traffic management efficiency. On the other hand, the research results have promoted the application of computer vision technology in the unmanned field, provided more reliable solutions for traffic monitoring, target identification and other tasks of drones in complex environments, and promoted the further development of unmanned technology. In addition, the algorithm proposed in this study reduces the demand for computing resources while maintaining high performance, which is conducive to reducing the hardware cost of intelligent transportation systems and unmanned equipment, and improving practicality and generalization. In the future, with the continuous improvement and optimization of relevant technologies, the results of this research are expected to play a greater role in a wider range of traffic scenarios and meteorological conditions, and lay a solid foundation for building a more intelligent, safe and efficient transportation system.
\\
\\
\textbf{KEW WORDS:} UAV;  Target detection on foggy days;  Improve the YOLO algorithm;  CycleGAN Net;  Transformer Block
\end{enabstract}
\pagebreak