\section{总结与展望\label{结论}}

\subsection{总结}

本论文针对雾天环境下交通小目标检测这一具有重要现实意义的课题展开了深入研究,取得了显著的研究成果。课题背景方面,随着智能交通系统和无人技术的快速发展,交通物体检测对于保障道路安全和优化交通管理至关重要。然而,雾天等劣天气条件严重影响了无人机拍摄图像的质量,导致基于良好天气图像训练出的检测模型性能大幅下降,难以满足实际应用需求,尤其对于小目标检测而言,挑战更为严峻。

在研究现状分析中,论文全面阐述了目标检测算法的演变历程,从早期依赖人工特征的传统方法到基于深度学习的双阶段、单阶段算法的兴起,如 R-CNN 系列、Fast R-CNN、Faster R-CNN 以及 YOLO 系列等,指出了各自的优势与局限性。对于图像去雾还原算法,论文深入探讨了基于物理模型的传统方法和基于深度学习的创新方法,如基于 GAN、Transformer 架构的去雾算法,分析了不同方法在雾天成像原理、先验知识利用以及特征学习等方面的差异与进展。同时,论文还对雾天目标检测技术进行了梳理,指出了现有方法在去雾与检测算法融合优化方面的不足。

论文的研究内容紧密围绕雾天环境下交通小目标检测问题展开,重点对图像去雾和目标检测算法进行了改进与优化。在目标检测方面,以 YOLOv11 算法为基础,提出了改进的 EX-YOLO 算法。通过对其网络结构的深入分析,发现 YOLOv11 在交通小目标检测中存在浅层特征图分辨率不足、跨尺度特征融合效率低等问题。为此,研究团队引入了改进的特征融合模块 SPPC,融合 SPPF 模块的多尺度特征提取能力和 CAM 的特征增强能力,显著提升了对小目标的特征捕获能力。同时,采用轻量化的卷积模块 DBSS 替代原始模块,降低了计算复杂度,使其更适用于资源受限设备。此外,引入 NWD 损失函数与 DIoU 损失函数相结合,提高了对小目标定位的精准度。

在图像去雾方面,论文基于 CycleGAN 网络提出了改进的 CGANFormer 算法。针对传统去雾方法在不同雾天场景下适用性和稳定性不足的问题,将 Transformer 模块与 CycleGAN 的生成器网络相结合,利用其自注意力机制捕捉图像全局依赖关系,突破了传统卷积局部感受野的限制,能够更精准地还原雾天图像中的细节信息,如物体边缘和纹理。同时,采用局部 - 全局结合的判别器架构,局部判别器关注图像细节特征,全局判别器把控图像宏观特征,提升了去雾图像的整体质量和视觉效果。

实验部分,论文在多个数据集上对所提出的算法进行了全面验证,包括 TT100K、VisDrone、FOG-TT100K 和 FOG-VisDrone 等,涵盖了不同的交通场景和雾天条件。实验结果表明,EX-YOLO 算法在保持较高检测精度的同时大幅降低了计算量,与 YOLOv11s、YOLOv10s 和 YOLOv9s 等基准模型相比,在 mAP、P、R 等关键指标上均展现出优异的综合性能,尤其在小目标检测任务中具有很强的竞争力,有效平衡了精度与效率,满足了无人机等有限设备环境下的实际应用需求。消融实验进一步证实了各改进模块的有效性,如 NWD 损失函数提高了模型对小目标的检测性能,SPPC 模块增强了特征提取和识别能力,DBSS 模块在减少计算量的同时保持了检测性能。

在雾天图像去雾和目标检测联合实验中,CGANFormer 与 EX-YOLO 算法的结合在 FOG-TT100K 和 FOG-VisDrone 数据集上的表现同样出色,相比其他先进算法,在精确度、召回率和 mAP 等指标上均取得显著提升,同时计算量显著降低,展现出较低的计算复杂度,证明了该算法在无人机等资源受限设备上进行实时目标检测的可行性,有效解决了雾天环境下交通目标检测的难点问题,为复杂雾天场景下的交通目标检测提供了高效、鲁棒的解决方案。


\subsection{展望}

尽管本论文在雾天环境下交通小目标检测方面取得了显著的研究成果,但仍存在一些可以进一步改进和拓展的方向。未来的研究可以从以下几个方面展开:

一方面,进一步优化算法的性能和效率。虽然 EX-YOLO 和 CGANFormer 算法在平衡检测精度和计算效率方面取得了良好效果,但随着智能交通系统的不断发展和应用场景的日益复杂,对于实时性和准确性的要求也在不断提高。因此,可以继续探索更先进的网络架构和优化策略,例如结合模型压缩技术、神经网络架构搜索(NAS)等方法,在保持甚至提升检测性能的同时,进一步降低模型的计算量和参数量,以适应更高分辨率图像和更快帧率的实时检测需求,为无人机在复杂交通环境中的大规模应用提供更强大的技术支持。

另一方面,加强多模态数据融合的研究。目前的研究主要集中在基于可见光图像的目标检测和去雾,然而在实际的智能交通场景中,还可以获取其他模态的数据,如红外图像、毫米波雷达数据等。这些不同模态的数据包含了互补的信息,例如红外图像对于目标的热辐射特性敏感,毫米波雷达能够穿透雾霭获取目标的距离和速度信息等。未来的工作可以探索如何有效地融合多模态数据,将其与可见光图像的优势相结合,构建更加鲁棒和全面的雾天交通目标检测系统,提高对复杂交通场景的理解和感知能力,为智能交通决策提供更丰富的信息依据。

此外,提升模型的泛化能力和适应性也是重要的研究方向。尽管通过在多个数据集上的实验验证了算法的有效性,但在实际应用中,交通场景和雾天条件的多样性以及光照、天气等环境因素的动态变化,可能会对模型的性能产生影响。因此,需要进一步研究如何增强模型在不同地域、不同季节、不同时间段等多样化场景下的泛化能力,使其能够快速适应各种复杂环境,减少对特定数据集的依赖。这可能涉及到无监督域适应、自监督学习等方法的应用,以及对数据增强技术的深入探索,通过模拟更广泛的场景变化和雾天条件,提高模型的鲁棒性和通用性,从而在实际的智能交通系统中实现更广泛、更稳定的应用。

本论文的研究为雾天环境下交通小目标检测提供了新的思路和有效的方法,具有重要的理论意义和应用价值。在未来的研究中,通过进一步优化算法、融合多模态数据以及提升模型的泛化能力等手段,有望推动该领域的研究向更深入、更实用的方向发展,为智能交通系统和无人技术在复杂气象条件下的广泛应用奠定更加坚实的基础。
