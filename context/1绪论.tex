\section{绪论\label{绪论}}

\subsection{课题背景及研究的目的和意义}

随着智能交通系统和无人技术的迅猛发展,交通物体的检测和识别已成为确保道路安全和优化交通管理的关键环节。大量研究表明,这一技术对于构建高效、安全的现代交通体系具有不可替代的重要作用 \cite{review1, review2, review3}。近年来,无人机图像目标检测技术取得长足进步,一系列创新方法不断涌现\cite{ye2022dense, Mffsodnet, sun2022rsod},这些成果为无人机图像目标检测领域带来了显著的性能提升,使其在良好天气和充足光照条件下的检测效果达到令人满意的水平,为实际应用奠定了坚实基础。

然而,实际的交通环境远比理想条件复杂多变,特别是在劣天气环境下,无人机拍摄图像的质量会受到严重干扰和影响。其中,雾天环境对图像质量的破坏尤为突出,雾气的弥散作用导致图像能见度大幅降低、对比度显著下降,目标轮廓变得模糊不清,边界难以辨认。这种情况下,基于良好天气图像训练出的检测模型往往难以适应,准确识别目标的能力受到极大限制,检测性能出现明显下滑,无法满足实际应用中对高精度检测的需求。

小目标检测在无人机交通物体检测场景中同样面临诸多严峻挑战。对于交通物体检测而言,小目标如远处的行人、交通标志等,由于其体积小、在复杂背景中易被遮挡,在图像中所占像素比例极低。传统检测算法受限于分辨率和信息丢失问题,难以有效捕捉和识别这些小目标,检测效果不尽如人意,进而影响整个智能交通系统的可靠性和稳定性。尤其在夜间和恶劣天气等复杂交通环境下,对检测算法的鲁棒性提出了更高的要求。此时,准确识别行人和其他交通参与者变得尤为重要,这不仅关乎智能交通系统的安全运行,更是无人驾驶技术得以进一步发展的关键前提,任何微小的检测失误都可能引发严重的后果。

在雾天环境下,小目标检测的难度被进一步放大。雾天图像质量的退化使得原本就难以识别的小目标更加模糊不清,目标的特征信息被大量遮蔽和混淆,特征提取过程变得更加困难重重。尽管一些研究开始关注这一棘手问题,并尝试提出相应的改进方法,例如通过结合图像去雾技术和目标检测算法,先对雾天图像进行预处理,增强图像质量,改善能见度和对比度,再进行目标检测,从而在一定程度上提高了检测性能。但目前这些方法仍存在诸多亟待解决的挑战,如去雾过程可能导致目标细节的丢失,影响检测精度;不同去雾算法在不同雾天场景下的适用性和稳定性存在差异;去雾与检测算法的融合优化尚未达到理想状态,两者的协同作用未能充分发挥,导致整体检测系统的性能提升有限。

雾天环境和小目标特性的复杂组合给无人机图像目标检测带来了前所未有的难题。如何有效整合图像去雾、图像增强技术与先进的深度学习检测算法,形成一套高效、鲁棒的检测系统,以提高在雾天环境下对小目标的检测能力,已成为当前计算机视觉和交通流量管理领域亟待攻克的重要研究课题。解决这一难题对于提升无人机在实际复杂环境中的应用效果、推动智能交通系统和无人驾驶技术的发展具有极其重要的理论意义和应用价值,有望为未来智能交通体系的构建提供坚实的保障。


\subsection{国内外研究现状}

\subsubsection{目标检测算法}

目标检测作为计算机视觉领域的核心任务之一,在众多实际应用场景中发挥着关键作用,如自动驾驶、智能安防、医疗影像分析、机器人视觉等。它的目标是在图像或视频中准确识别出特定目标物体的位置和类别。随着计算机技术的飞速发展,目标检测技术经历了从简单到复杂、从低效到高效的演变过程,其理论和方法不断推陈出新,以满足日益增长的实际应用需求。

早期目标检测依赖人工设计的传统特征,如 Haar \cite{Haar_like} 特征,其能捕捉图像边缘等信息,通过积分图快速计算特征值,基于此的 Viola - Jones 人脸检测算法借助 AdaBoost 算法选择训练特征构建级联分类器实现高效面部检测。还有基于边缘、纹理特征等的方法,这类传统特征提取方式虽直观、计算复杂度低,但对复杂场景适应性差,难以自动学习深层规律,对目标各种变化敏感致检测精度受限。
后机器学习发展,浅层学习方法如利用 SVM 分类的目标检测受关注\cite{svm},通常先用 SIFT、HOG 等特征提取手段获取目标特征再用分类器区分目标与背景。其中 HOG 特征描述子在行人检测等任务表现好,但浅层学习方法在特征提取阶段仍有限,特征表达能力不足,难全面刻画复杂目标特征,且训练需大量标注数据,面对新类别、场景泛化能力弱。

深度学习的兴起为目标检测带来了革命性的变化。深度卷积神经网络(CNN)\cite{cnn}能够自动学习图像中的深层特征表示,从而极大地提高了目标检测的性能。
早期的基于深度学习的目标检测方法之一是 R-CNN\cite{fast_rcnn,mask_rcnn}。它首先通过选择性搜索等区域提议方法生成候选区域,然后对每个候选区域进行特征提取和分类。然而,R-CNN 存在计算速度慢、训练过程复杂等缺点,因为它对每个候选区域都要进行独立的 CNN 计算。

为了克服这些问题,Fast R-CNN 相继提出。Fast R-CNN 将整个图像输入网络一次性提取特征,然后对所有候选区域共享这些特征,大大提高了计算效率。之后,Faster R-CNN 进一步引入了区域提议网络(RPN)来替代传统的区域提议方法,实现了端到端的训练,进一步提升了检测速度和精度,成为目标检测领域的一个重要里程碑。

随后,YOLO(You Only Look Once)\cite{yolov1, yolov2, yolov3, yolov4, yolov6, yolov7, yolov9, yolov10, yolov11}系列算法和 SSD(Single Shot Multibox Detector)\cite{ssd, dssd}等单阶段目标检测算法崭露头角。YOLO 将目标检测任务视为一个回归问题,将图像划分为多个网格,每个网格同时预测多个边界框和类别概率,实现了快速、实时的目标检测,在一些对速度要求较高的应用场景中表现出色。

YOLOv1 以提升检测速度为核心,将检测任务视为回归问题求解,不过在准确性方面尚有不足。YOLOv4 在此基础上引入 Mish 激活函数、CSPDarknet 主干网络\cite{csp}及 SPP 模块等先进组件,显著提升检测精度与速度。YOLOv9 首次引入混合 CNN - Transformer 主干网络架构,并融合可编程梯度信息(PGI)与广义高效层聚合网络(GELAN),使小目标检测性能实现质的飞跃。YOLOv10 创新性地引入无 NMS训练机制与多尺度特征融合技术,进一步优化小目标检测性能。YOLOv11 则通过引入 C3k2 和 C2PSA 模块,对网络结构与训练策略进行深度优化,再次提升检测精度与速度。这些持续迭代的改进,不仅全方位提升模型检测精度与速度,还显著增强其在复杂场景下的稳健性与适应性,为无人机小目标检测等前沿应用提供坚实可靠的技术支撑。

然而,尽管 YOLO 的单级结构在效率上独树一帜,小目标检测仍面临诸多挑战。小目标像素稀缺,极易受背景干扰,致使检测准确性下降。FPN\cite{fpn}应运而生,其借助自上而下的功能融合机制,有效提升多尺度检测能力。

在损失函数优化领域,传统 YOLO 采用的二进制交叉熵(BCE)与均方误差(MSE)\cite{mse}对小目标检测中边界细微变化敏感度较低,易引发准确性降低。IoU 损失函数系列,包括 GIoU\cite{giou}、DIoU 与 CIoU\cite{diou},通过对预测框与真实框重叠区域优化,显著提高检测精度。
Yang 等人\cite{gwd}提出的 GWD(Generalized Wasserstein Distance)损失函数,创新性地通过位置关系测量边界框,提升定向物体检测准确性,但对小目标的敏感度仍有待提高。

为攻克小目标检测难题,研究人员致力于优化 FPN 结构。
特征金字塔网络(FPN)开创性地整合不同尺度特征,增强对各类目标的检测能力。
FPN 由 Tsung - Yi Lin 等人于 2017 年首次提出,通过自上而下路径在特征图上取样,巧妙融合高级语义信息与低级空间信息,使其在小目标检测领域表现卓越。PANet(Path Aggregation Network)\cite{pan}与 BiFPN(Bidirectional Feature Pyramid Network)\cite{bifpn}等网络在 FPN 基础上推陈出新。
PANet 增加自下而上的路径聚合与全景特征融合,助力浅层信息向深层高效传输,适用于复杂背景下的小目标检测。
BiFPN 则借助双向特征融合与权重学习机制,自适应调整特征层贡献,进一步优化多尺度检测效果。
Su Peng 等人\cite{mod-yolo}设计的 GRF - SPPF 模块,通过整合全球与本地传感现场信息,有效降低不同尺度对检测的影响。
Li Haibin 等人\cite{yolo-pl}提出的 E - PAN(Enhanced Path Aggregation Network),通过相同尺度近似采样与残差操作,强化多尺度特征融合,过滤复杂背景信息以提升检测精度。
Zhao Chao 等人\cite{rdd-yolo}设计的双特征金字塔网络(DFPN),增强特征融合能力,提高整体检测性能。
Zhang Yan 等人\cite{dsp-yolo}打造的轻量级且对细节敏感的 DsPAN,强化本地特征与细节,减少特征融合过程中的信息损失。
然而,深度学习算法在追求高精度过程中,通常伴随着计算开销的显著增加,在有限计算资源约束下,实现实时检测面临巨大挑战。


\subsubsection{图像去雾还原算法}

图像去雾是计算机视觉关键技术之一,在安防监控、自动驾驶等众多实际应用场景中,大气散射常导致图像退化,出现对比度降低、颜色失真、细节模糊等问题,影响后续图像分析与识别任务准确性,深入研究图像去雾还原算法技术理论发展及现状意义重大。

20 世纪 80 年代,大气光学相关研究为图像去雾奠定基础,大气散射模型被提出,阐述光线传播受散射和吸收作用机制,因当时假设大气是均匀介质且对不同波长光线散射作用相同,模型较简单,后续研究引入非均匀大气介质假设,考虑不同波长光线散射特性差异,使模型更贴近实际。模型可表示为:
\begin{equation}
    \label{eq:haze}
    I(x) = J(x)t(x)+A(1-t(x))
\end{equation}

公式 \ref{eq:haze} 中,$I(x)$ 表示观测到的含雾图像,$J(x)$ 是无雾场景的清晰图像,$t(x)$ 为大气透射率,反映了场景中光线传播时未被散射和吸收的比例,$A$ 是大气光照,在远处场景点,大部分光线被散射,此时观测到的像素值趋近于大气光照。

20 世纪末至 21 世纪初,随着计算机视觉技术发展,研究者利用图像先验知识去雾,物理模型去雾方法优势凸显,其中基于暗通道先验方法地位重要,其通过计算暗通道图、估算透射率并与大气散射模型结合去雾。如 Dark-ControlNet \cite{Dark_ControlNet}融合冻结骨干网络与暗通道先验特征提升去雾效果,He 等人\cite{he2010single}提出的方法结合暗通道先验与雾霾成像模型恢复无雾图像,Peng 等人\cite{peng2018generalization}引入自适应色彩校正机制应对图像退化问题。

基于颜色衰减先验的方法为物理模型去雾提供了另一条有效途径,其关键在于利用雾气影响下图像亮度与饱和度之间的内在关系来实现去雾。Zhu 等人 \cite{zhu2015fast} 提出的基于颜色衰减先验的单图像去雾方法,通过构建线性模型精准估计场景深度,并借助监督学习优化参数,从而高效恢复出透射率和场景光芒,实现雾霾的有效去除。Liang 等人 \cite{liang2021single} 设计的基于衰减图引导的色彩校正方法以及细节保留的去雾方法,在色彩校正和细节保留方面表现出色。Qiu 等人 \cite{qiu2024perception} 对物理模型进行改进,创新性地结合超像素场景先验(SPSP)和简单线性迭代聚类(SLIC)等前沿技术,对去雾过程进行了深度优化。

随着人工智能技术的飞速发展,深度学习在图像去雾领域逐渐崭露头角,成为当下图像去雾的主流方向。众多基于深度学习的去雾方法不断涌现并得到广泛验证,其中包括基于生成对抗网络(GAN)、循环神经网络(RNN)以及 Transformer 架构的图像去雾方法。Dong 等人 \cite{dong2020fd} 提出的带有融合判别器的端到端生成对抗网络,用于单张图像去雾,其巧妙地以频率信息作为先验知识,有效解决了传统去雾方法中因中间参数估计不准确而引发的伪影和颜色失真问题。Ashwini 等人 \cite{ashwini2024epq} 采用差分进化算法,并创新性地结合感知损失、质量评估损失和对抗性损失对对抗生成网络进行训练,进一步提升了去雾效果。Zheng 等人 \cite{zheng2022dehaze} 提出的增强型注意力引导生成对抗网络,专门针对遥感图像去雾难题,取得了显著成效。Frants 等人 \cite{frants2023qcnn} 提出的基于鲁棒性四元数神经网络架构的单图像去雾方法,为去雾技术提供了新的思路。Guo 等人 \cite{guo2022image} 通过引入特征调制和传输感知 3D 位置嵌入模块,成功解决了 CNN 和 Transformer 之间存在的特征不一致问题。Song 等人 \cite{song2023vision} 对 Swin Transformer 的归一化层(如用 RescaleNorm 替代 LayerNorm)、激活函数(ReLU 优于 GELU)和空间信息聚合方案进行了全面改进,进一步提升了深度学习去雾方法的性能。

\subsubsection{雾天目标检测技术}

在复杂环境下的目标检测领域,雾天场景下的目标识别占据着举足轻重的地位。传统雾天目标检测技术的流程通常是,先借助图像恢复技术,像去雾算法这类手段,对雾天图像开展预处理操作,以此来提高图像的视觉质量。紧接着,将经过去雾处理的图像输入至预先训练好的目标检测模型之中进行识别工作。例如,部分研究通过引入图像增强技术,包括对比度调整或者基于物理模型的散射估计方法等,来提升雾天图像的可见性。与此同时,还有一些研究着重于设计具备鲁棒性的目标检测算法,以便能够适应去雾之后图像的特征变化情况。这种分阶段处理的模式,虽能在一定程度上减轻雾天环境对于目标检测性能产生的负面干扰,然而其存在的局限性不容忽视,即图像恢复与目标检测任务之间缺乏协同优化机制。去雾算法一般是独立于检测任务来进行优化的,这就造成生成的去雾图像可能无法充分契合目标检测对于特征表达的特有需求,进而在很大程度上限制了整体性能的进一步提升。

基于此现状,近年来研究的重点方向逐渐向联合优化去雾与目标检测这一领域倾斜\cite{liu2024oriented, liu2024approach, zhang2020unified},也就是所谓的检测友好的去雾方法。此类方法不再仅仅执着于实现高质量的图像恢复,转而致力于优化去雾过程,使其能够与检测任务的特征学习相互适配。

在诸如雾天、雨天这类降级环境下,目标特征往往会被掩盖或者模糊化,倘若直接在降质图像上开展目标识别,其效果通常都不尽如人意\cite{freire2024beyond}。不过,图像恢复技术,像去雾、去噪等操作,能够有效改善图像质量,进而增强目标识别的性能表现。目前的研究大多将图像恢复与目标识别分割开来,主要聚焦于图像质量的评价,而鲜少有研究着重关注恢复后的图像在实际目标识别任务中所实现的性能提升情况。本研究着重探究图像恢复与目标识别之间存在的内在联系,并通过实验来验证图像恢复技术对于目标识别性能产生的实际影响。

部分研究选择直接在降质图像上开展目标识别训练工作,通过改进对比损失、利用辅助信息以及先验知识等手段来增强模型的性能。例如,Singh 等人\cite{singh2019dual}提出 DirectCapsNet 双导向胶囊网络模型,经由引入 HR-anchor 损失以及目标重建损失,借助高分辨率图像辅助训练,从而显著提升了低分辨率(VLR)图像的识别性能。Sindagi 等人\cite{sindagi2020prior}构建了一种基于先验知识的无监督领域自适应目标检测框架,通过引入先验对抗损失以及残差特征恢复块。Zhong 等人\cite{zhong2024dehazing}设计出先验知识引导网络,依托大气散射模型以及共现关系图来引导特征学习。

而另外一些研究则是先训练用于恢复图像的模型,将降质图像恢复之后再输入至模型中进行检测。这类方法普遍分为恢复与识别两个阶段,先训练图像恢复算法,随后在增强后的图像上对预训练的对象识别算法开展评估工作。由于恢复和识别任务相互分离,实验较难验证它们之间存在的相互作用情况。例如,Hu 等人\cite{hu2024beyond}提出名为 HRAOD 的雾霾稳健航空目标检测方法,引入图像去雾方法作为目标检测的预处理步骤。而少数研究将去雾模块同检测模块进行联合训练,像 Liu 等人\cite{liu2024oriented}提出遥感图像目标检测模型 DFENet,引入去雾特征增强模块以及动态平衡机制。Liu 等人\cite{liu2024approach}提出一种轻量化目标检测模型,实现与去雾模块的联合优化。Zhang 等人\cite{zhang2020unified}以雾霾密度为先验知识,经由残差感知雾霾密度分类器、密度感知去雾网络以及密度感知目标检测器开展相关工作。

与上述各类方法存在差异的是,本文同时将目光聚焦于图像的质量恢复以及恢复后图像的目标检测效果方面。通过预先训练改进后的生成器和判别器,以此确保图像质量恢复的效果得以保障,同时运用该网络来制作模拟雾天场景下的数据集。将预训练完成的生成器与改进后的目标检测进行联合优化,使得生成器所生成的去雾图像能够更加契合目标检测任务的需求,进而进一步提升了检测性能表现。


\subsection{论文研究的主要内容及组织架构}

\subsubsection{论文研究内容}

在雾霾天气条件下,由于特定天气信息的干扰,所拍摄的图像能见度较低,尤其在交通路口,目标数量众多且种类相对单一,呈现出目标物密集、尺度多样化等特性。在低能见度的雾天环境下,监测采集的图像不仅视觉效果欠佳,还存在噪声较高的问题,致使目标检测的难度显著增大。

与一般目标检测领域相比,雾天环境下交通路口的目标检测识别面临更大的挑战。行人、车辆以及交通标志的检测工作面临诸多困难,尤其是图像中远处尺寸较小的交通标志,其特征提取难度较大。为应对这一复杂情况,本文以图像去雾和不同尺寸目标检测作为研究重点,对相关算法进行了改进与优化,致力于提升雾霾天气下行人、车辆以及交通标志检测算法的效能。

针对雾天图像去噪问题,本文提出基于对抗生成网络的去雾方法,旨在生成干净清晰的图像。我们构建了基于 CycleGAN \cite{cgan}的去雾网络 CGANFormer,该网络采用全局 - 局部鉴别器结构,能够有效应对空间变化的雾霾天气状况。同时,为更好地保存细节信息,设计了融合 Transformer 模块与残差块连接的编码 - 解码生成器架构,以生成高质量的无雾图像。

区别于传统的依赖大气散射模型参数估计的方法,本文通过定义一种颜色损失,并将其与感知损失相结合,共同构成 CycleGAN 中的损失函数。这一改进措施不仅提高了纹理信息恢复的质量,还能够生成具有鲜艳颜色的图像,有效缓解了颜色失真的问题。

为了提升对不同尺寸行人、车辆以及交通标志的检测能力,并兼顾精确度与实时性,本文选择当前性能优良的 YOLOv11 算法作为改进基础,新算法名为 EX-YOLO。在 YOLOv11 的 SPPF 模块基础上,我们改进为 SPPC 模块,以学习更深刻的多重尺度目标信息,特别是中小目标的信息。考虑到实际应用中计算资源的限制,尤其是无人机等设备的运行需求,我们引入轻量化的卷积模块,确保算法在有限计算资源下能够高效运行。

此外,采用 DIoU 损失和 NWD 损失函数\cite{nwd},使得预测框和对小目标的识别能力更为精准和激进。这些改进措施在不显著增加计算成本的前提下,有效提升了算法在多尺度目标识别方面的能力。

本文将 CGANFormer 与改进后的 EX-YOLO 算法进行联合应用。在去雾网络的训练过程中,我们使用 NYU2\cite{nyu2}、Dense-Haze\cite{NTIRE_Dehazing_2019} 数据集进行训练。
同时,利用 TT100K\cite{tt100k}、VisDrone\cite{vd} 数据集和FOG-TT100K、FOG-VisDrone开展去雾目标检测网络的联合训练。通过精心设计的实验方案,我们对所提出的算法与不同目标检测算法进行对比实验,并开展消融实验,以系统地验证本文算法的有效性与优势。

综上所述,本文的研究内容紧密围绕雾霾天气下交通路口的目标检测问题,通过图像去雾与目标检测算法的改进与联合,力求为该领域的研究提供新的思路与方法,最终实现对行人、车辆以及交通标志等目标的高效、准确检测。

\subsubsection{论文章节安排}

第一章是绪论。
首先聚焦于雾天下无人机交通物体检测这一前沿领域,系统梳理了当前研究进展与现状,精准识别面临的关键问题与挑战。
在深度汲取与融合前人研究成果的基础上,创新性地提出了一套改进方案,旨在优化图像去雾检测算法,构建兼具高效性与健壮性的雾天交通物体检测技术方案。
最后,对整体研究内容与架构进行了系统概述,为后续各章节的深入探究及整个研究流程的有序推进提供了有力支撑。

第二章是相关理论基础。
首先系统性地阐述了目标检测的理论基础与算法体系,涵盖了从传统机器学习方法到基于深度学习的单阶段与双阶段目标检测算法。然后,聚焦于图像去雾还原领域,详细解读雾天成像原理,探讨基于深度学习的图像去雾算法的创新突破。
最后,对目标检测算法及图像去雾还原算法的评价指标进行了分析,为后续实验验证与算法性能比较提供了坚实的理论支撑与量化依据。

第三章是基于 YOLOv11 网络改进的交通小目标检测算法。
首先对 YOLOv11 网络结构进行了阐述,详尽分析了其各个模块所承载的作用,深入剖析了该网络在目标检测任务中的运作机制。
然后介绍了改进后的 EX-YOLO 网络结构,对其新增及优化的各个模块予以细致解读,明晰各模块如何协同作业以提升网络性能。
最后,通过严谨的实验设计,对 EX-YOLO 网络与其他现有多项主流网络的性能展开了全面且系统的对比分析,借助多维度的评估指标,直观呈现了 EX-YOLO 网络在目标检测领域的显著优势与优越性能表现,有力论证了改进方案的有效性与合理性。

第四章基于 CycleGAN 网络改进的去雾还原算法。
首先系统阐述了 CycleGAN 网络的运行原理,剖析了生成器与判别器的架构细节,全面展现了二者在对抗训练中的协同机制。然后聚焦于 Transformer 模块的运作机制,凭借其自注意力机制在捕捉长程依赖关系方面的优势,创新性地提出了将 Transformer 融入 CycleGAN 构成的改进网络 CGANFormer,详尽解读了改造后的网络架构如何重塑特征学习与图像映射能力。
最后,通过严谨的实验设计,验证了 CGANFormer 在图像去雾任务上相较于其他去雾网络的卓越性能。更进一步,本研究开创性地将 CGANFormer 与 EX-YOLO 目标检测网络有机结合,构建出一种全新的目标检测架构,使其在有雾或无雾天气条件下达成更精准高效的目标检测效果,为复杂环境下的视觉智能应用提供了全新的技术范式与性能突破路径。

第五章是结论与展望。
对本文中的研究内容及其结果做出总结归纳,然后分析雾天下无人机交通物体检测任务存在的不足与未来可以进一步研究的工作。

\subsection{本章小结}

本章深入阐述了在雾天环境下无人机交通物体检测领域的关键挑战与研究进展。首先,剖析了雾天环境对图像质量的负面影响,以及小目标检测在复杂交通场景中的特殊难点,明确了提升检测性能的迫切需求。接着,全面综述了目标检测算法的发展历程,从传统特征提取方法到基于深度学习的先进架构,重点关注了 YOLO 系列算法及其改进措施在小目标检测中的应用;同时,详细介绍了图像去雾技术的演变,包括传统物理模型和新兴的深度学习方法,探讨了去雾与目标检测融合的前沿研究方向。最后,本章概述了论文的研究内容与架构,提出通过优化图像去雾和目标检测算法,构建高效、鲁棒的雾天交通物体检测系统,为后续章节的具体算法改进与实验验证奠定了坚实的理论基础。
