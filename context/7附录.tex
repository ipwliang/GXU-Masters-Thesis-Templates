\begin{appendices}
% \appendix
\setcounter{table}{0}
\setcounter{figure}{0}
\setcounter{equation}{0}
\renewcommand{\thetable}{\thesection-\arabic{table}}
\renewcommand{\theequation}{\thesection-\arabic{equation}}
\renewcommand{\thefigure}{\thesection-\arabic{figure}}

\ctexset{section={name={附录},number=\thesection}}%修改章节序号以符合附录
    
\section{第一个附录\label{附录:第一个附录}}
周后稷,名弃。其母有邰氏女,曰姜原。姜原为帝喾元妃。姜原出野,见巨人迹,心忻然说,欲践之,践之而身动如孕者。居期而生子,以为不祥,弃之隘巷,马牛过者皆辟不践;徙置之林中,適会山林多人,迁之;而弃渠中冰上,飞鸟以其翼覆荐之。姜原以为神,遂收养长之。初欲弃之,因名曰弃。

弃为兒时,屹如巨人之志。其游戏,好种树麻、菽,麻、菽美。及为成人,遂好耕农,相地之宜,宜穀者稼穑焉,民皆法则之。帝尧闻之,举弃为农师,天下得其利,有功。帝舜曰:“弃,黎民始饥,尔后稷播时百穀。”封弃於邰,号曰后稷,别姓姬氏。后稷之兴,在陶唐、虞、夏之际,皆有令德。

\pagebreak %强制结束此页,也可以使用\newpage代替


\section{第二个附录\label{附录:第二个附录}}
崇侯虎谮西伯於殷纣曰:“西伯积善累德,诸侯皆向之,将不利於帝。”帝纣乃囚西伯於羑里。闳夭之徒患之。乃求有莘氏美女,骊戎之文马,有熊九驷,他奇怪物,因殷嬖臣费仲而献之纣。纣大说,曰:“此一物足以释西伯,况其多乎!”乃赦西伯,赐之弓矢斧钺,使西伯得征伐。曰:“谮西伯者,崇侯虎也。”西伯乃献洛西之地,以请纣去砲格之刑。纣许之。

西伯阴行善,诸侯皆来决平。於是虞、芮之人有狱不能决,乃如周。入界,耕者皆让畔,民俗皆让长。虞、芮之人未见西伯,皆惭,相谓曰:“吾所争,周人所耻,何往为,祇取辱耳。”遂还,俱让而去。诸侯闻之,曰“西伯盖受命之君”。

\end{appendices}
